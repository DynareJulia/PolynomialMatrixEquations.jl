\documentclass{article}
\usepackage{times}
\usepackage{amsmath}
\usepackage{amssymb}


\title{Solving matrix equation $A_2X^2 + A_1X + A_0 =0$}
\author{Michel Juillard}
\date{December 2022}

\begin{document}
\maketitle

We are interested in real solutions to equation
\[
  A_2X^2 + A_1X + A_0 =0
\]
where $A_i,\;\;\; i=0,1,2$ are square matrices of order $n$ and we
want to know if there exist a unique solution where the eigenvalues of
the solution $X$ are inside the unit circle.

We consider two algorithms: cyclic reduction and generalized Schur
decomposition.

We can take advantage of the zero columns of $A_i,\;\;\;
i=0,1,2$. Let's define $\iota_i$ as the set of indices of non-zero columns of matrix
$A_i$ and $\iota_b$, the intersection of $\iota_0$ and $\iota_2$,
corresponding to the column that aren't zero in $A_0$ and in
$A_2$. The set $\iota_m = \iota_0 \setminus \iota_b$ indicates the columns
that are non-zero in $A_0$ but not in $A_2$. $b_{\iota_0}$ collects
the indices of the element of $b$ in $\iota_0$ and $b_{\iota_2}$ collects
the indices of the element of $b$ in $\iota_2$. 

We note $M^{c}$ a matrix made of a selection of its columns in set $c$
and $M^{(r, c)}$ a submatrix made of rows in set $r$ and columns in
set $c$. 

\section{Generalized Schur decomposition}
Before using generalized Schur decomposition, we need to transform the
original problem in a linear one. If
\[
  A_2X^2 + A_1X + A_0 =0
\]
we can rewrite the equation as
\[
  \begin{bmatrix}
    A_1 & A_2\\
    I & 0
  \end{bmatrix}
  \begin{bmatrix}
    I \\
    X
  \end{bmatrix}
  X
  =
  \begin{bmatrix} 
    -A_0 & 0\\
    0 & I
  \end{bmatrix}
  \begin{bmatrix}
    I \\
    X
  \end{bmatrix}
\]
or, taking into account the empty columns in $A_i,\;\;\; i=0,1,2$
\[
  \begin{bmatrix}
    A^{(\iota_m)}_1 & A^{(\iota_2)}_2\\
    I^{(b_{\iota_m})} & 0
  \end{bmatrix}
  \begin{bmatrix}
    I \\
    X^{(\iota_2,\iota_0)}
  \end{bmatrix}
  X^{(\iota_0,\iota_0)}
  =
  \begin{bmatrix} 
    -A^{(\iota_0)}_0 & -A^{(\iota_2)}_1\\
    0 & I^{(b_{\iota_2})}
  \end{bmatrix}
  \begin{bmatrix}
    I \\
    X^{(\iota_2,\iota_0)}
  \end{bmatrix}
\]
where $I$ matrices are conformant. $I^{(b_{\iota_i})}$ is a matrix
with as many columns as $A^{(\iota_i}_i$ and made of unit vectors in
columns $b_{\iota_i}$ for $i=0,2$. Note that the choice of putting
columns $A^{\iota_b}_1$ on the right hand side is arbitrary but has no
implication on the results.

Let's note
\[
  D = 
  \begin{bmatrix}
    A^{(\iota_m)}_1 & A^{(\iota_2)}_2\\
    I^{(b_{\iota_m})} & 0
  \end{bmatrix}
\]
and
\[
  E =
  \begin{bmatrix} 
    -A^{(\iota_0)}_0 & -A^{(\iota_2)}_1\\
    0 & I^{(b_{\iota_2})}
  \end{bmatrix}
\]  

Then we can use the generalized Schur decomposition to select an $X$
matrix with all eigenvalues inside the unit circle and check that it
is unique.




\end{document}

%%% Local Variables:
%%% mode: latex
%%% TeX-master: t
%%% End:
